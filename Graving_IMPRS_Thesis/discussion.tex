\chapter*{Discussion}
\addcontentsline{toc}{chapter}{Discussion}
\markboth{Discussion}{Discussion}
Here I have discussed computer vision and deep learning-based methods for the study of animal behavior. I highlighted existing methods and introduced new advances in these methodological areas that helped to solve limitations of existing approaches. Some of the work presented here has already been widely adopted by the animal behavior community and has aided other researchers to make further advances in the field of behavioral quantification. However, this field of quantitative behavioral research is still in its infancy and the techniques presented in this thesis are only scratching the surface of the full set of approaches that are available for studying animal behavior. In the future, it is likely that even more general-purpose methods will be developed and applied to a range of experimental tasks. Considering the different use cases and applications that are highlighted in this thesis, deep learning and computational modeling will almost certainly become (and arguably already is) an important and powerful general-purpose tool for understanding animal behavior. 

The techniques presented in this thesis for measuring behavior have already evolved since their publication. For example, some of the methods have been made obsolete, or the general ideas have been built on and improved in subsequent work by other researchers, which only highlights the fast-moving nature of the field. The conventional 2D barcode tracking algorithms used in Chapter 1 are now out-of-date, and newer, more robust deep learning methods have been shown to be much more reliable for this task \citep{wild2018honeybee, boenisch2018tracking, sixt2018rendergan, hu2019deep}. Instead of using conventional computer vision to localize and decode barcodes, deep learning algorithms can be trained with artificially augmented data that better match real-world scenarios where conventional methods often fail \citep{sixt2018rendergan}. This approach includes simple techniques like using conventional methods to generate training data for deep learning algorithms, which can then be used to detect, localize, and decode barcodes \citep{hu2019deep}, as well as algorithms that generate completely artificial data in a way that matches the real-world image statistics of the task \citep{sixt2018rendergan}. There have also been recent advances in using deep learning for markerless recognition of individual birds \citep{ferreira2019deep}, where RFID tags are used in combination with inexpensive camera hardware to automatically generate training data for individual identification.

Other researchers working to advance individual tracking and pose estimation methods have begun to iterate on and apply the ideas proposed in Chapter 2. For example, newer methods have begun to integrate temporal information when training pose estimation models \citep{liu2020optiflex, wu2020deep} and have extended pose estimation methods to directly estimate posture and track pairs and groups of individuals directly from experimental videos without any of the preprocessing described in Chapter 2 \citep{pereira2020sleap}. However, many limitations still remain for these methods and there is a great deal of work left to be done. For example, generalizing to new experimental scenarios with limited training data \citep{mathis2020imagenet} and re-identifying individuals after visual occlusions both still remain exceptionally complex problems, especially in cases where the lighting or other environmental factors have changed dramatically \citep{romero2018idtracker, mathis2020imagenet}. The task of multiple pose estimation and pose tracking in 3D, especially in difficult field scenarios, still remains a challenging open problem as well. The extension of these algorithms to real-time and field-based experiments also poses issues; however, there has been recent progress on both of these fronts \citep{Zuffi:ICCV:2019, kane2020real}.

In general, the tools and methods presented in this thesis already provide ample opportunities to answer scientific questions, but the complexity of these data necessitates new tools to extract meaningful and interpretable information. In Chapter 3, I introduced general-purpose methods for interpreting and analyzing these behavioral data, however, this is only a first step and one of many possible ways to apply data-driven modeling for addressing important research topics. I discussed some of these challenges in Chapters 2 and 3, and this topic has also been discussed extensively elsewhere \citep{brown2018ethology, berman2018measuring, datta2019computational}. In particular, the methods I present in Chapter 3 can only model relatively short timescales and do not take into account longer timescales or individual differences in behavior. Future work will have to reckon with the conceptual challenges of modeling detailed behavioral data in a way that allows researchers to answer important scientific questions while also balancing model complexity with interpretability. Several approaches have been proposed for modeling behavior in recent years \citep{berman2014mapping, berman2016predictability, wiltschko2015mapping, Costa1501, johnson2016composing, markowitz2018striatum}, but all have their limitations. In the end, the complexities of modeling behavioral data enters the realm of the philosophical. However the central role of the scientist is to undertake these decisions for how best to model and interpret their data when making inferences about the natural world.
