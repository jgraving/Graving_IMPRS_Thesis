	\chapter*{Zussamenfassung}
    \addcontentsline{toc}{chapter}{Zussamenfassung}
  Das Studium des Verhaltens von Tieren ist ein grundlegendes Bestreben zur Beantwortung wissenschaftlicher Fragen in einer Vielzahl von Bereichen --- einschließlich Neurowissenschaften, Psychologie, Ökologie, Genetik und Evolution. Während die Aufgabe, genaue und vollständige Verhaltensdaten zu sammeln, in der Regel immer schwierig, mühsam und subjektiv war, gab es in den letzten Jahren rasche Fortschritte bei den Methoden zur automatischen objektiven und maßstabsgetreuen Quantifizierung von Verhalten. Dieser Fortschritt wurde in erster Linie durch das Aufkommen neuer rechnergestützter Hardware, Software und Algorithmen zur Verhaltensmessung vorangetrieben. Mit der zunehmenden Qualität und Auflösung dieser Daten steigt auch der Bedarf an neuen Methoden zur datengesteuerten Modellierung, die zentrale Erkenntnisse darüber liefern können, wie Tiere ihr Verhalten organisieren. In dieser Arbeit konzentriere ich mich auf die Entwicklung neuer Algorithmen und Software zur Messung und Modellierung von Verhalten mit Methoden des Deep-Learning, der Bayes'schen Inferenz und der probabilistischen Programmierung, wobei ich diese Ansätze auch mit Ideen aus anderen relevanten Bereichen wie Informationstheorie, nichtlineare Dynamik und statistische Physik zusammenführe. Zunächst entwickelte ich in Zusammenarbeit mit Kollegen ein Strichcode-Verfolgungssystem für automatisierte Verhaltensstudien, bei dem der Aufenthaltsort und die Identität von Individuen über längere Zeiträume mit herkömmlicher Computer-Sicht zuverlässig verfolgt werden kann (Kapitel 1). Als nächstes entwickelte ich Mehrzweck-Deep-Learning-Methoden für die Messung der Verhaltung von Tieren --- jede Menge vom Benutzer ausgewählter Körperteile --- im Labor und vor Ort (Kapitel 2). Schließlich stelle ich Methoden zur Verwendung dieser Haltungsdaten zur Verhaltensmodellierung mit Techniken des maschinellen Lernens und der statistischen Bayes'schen Inferenz vor (Kapitel 3). Zusammen verringern diese Methoden die Barrieren bei der Messung und Modellierung des Verhaltens von Tieren und ermöglichen es den Forschern, wissenschaftliche Fragen zu beantworten, die zuvor unlösbar waren.
