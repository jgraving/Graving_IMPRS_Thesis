	\chapter*{Zussamenfassung}
    \addcontentsline{toc}{chapter}{Zussamenfassung}
  Das Studium des Verhaltens von Tieren ist ein grundlegendes Unterfangen zur Beantwortung wissenschaftlicher Fragen in einer Vielzahl von Bereichen --- einschließlich Neurowissenschaften, Psychologie, Ökologie, Genetik und Evolution. Während die Aufgabe, genaue und vollständige Verhaltensdaten zu sammeln, in der Regel immer schwierig, mühsam und subjektiv war, gab es in den letzten Jahren rasche Fortschritte bei den Methoden zur automatischen objektiven und maßstabsgetreuen Quantifizierung von Verhalten. Dieser Fortschritt wurde in erster Linie durch das Aufkommen neuer rechnergestützter Hardware, Software und Algorithmen zur Verhaltensmessung vorangetrieben. Um zentrale Erkenntnisse darüber zu gewinnen, wie Tiere ihr Verhalten mit der verbesserten Qualität und Auflösung dieser Daten organisieren, werden neue Methoden zur datengesteuerten Modellierung benötigt. In dieser Dissertation konzentriere ich mich in dieser Arbeit auf rechnergestützte Werkzeuge - die Entwicklung neuer Algorithmen und Software - zur Messung und Modellierung von Verhalten unter Verwendung von Methoden aus den Bereichen Computersehen, Deep-Learning, Bayes'sche Inferenz und probabilistische Programmierung, wobei ich diese Ansätze auch mit Ideen aus anderen relevanten Bereichen wie Informationstheorie, nichtlineare Dynamik und statistische Physik zusammenführe. Zuerst entwickelte ich ein Strichcode-Verfolgungssystem für automatisierte Verhaltensstudien, bei dem der Aufenthaltsort und die Identität von Personen über mehrere Wochen (oder potenziell länger) mit konventionellem Computersehen zuverlässig verfolgt werden kann (Kapitel 1). Als Nächstes entwickelte ich Mehrzweck-Deep-Learning-Methoden für die Messung der Haltung von Tieren --- jede Menge vom Benutzer ausgewählter Körperteile --- im Labor und vor Ort (Kapitel 2). Schließlich stelle ich Methoden zur Verwendung dieser Haltungsdaten zur Verhaltensmodellierung mit Techniken des maschinellen Lernens und der statistischen Bayes'schen Inferenz vor (Kapitel 3). Zusammen verringern diese Methoden die Barrieren bei der Messung und Modellierung des Verhaltens von Tieren und ermöglichen es den Forschern, wissenschaftliche Fragen zu beantworten, die zuvor unlösbar waren.


