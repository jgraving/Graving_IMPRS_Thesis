	\chapter*{Summary}
	\markboth{SUMMARY}{SUMMARY}
	\addcontentsline{toc}{chapter}{Summary}
  The study of animal behavior is a fundamental pursuit for answering scientific questions across a variety of fields --- including neuroscience, psychology, ecology, genetics, and evolution. While the task of collecting accurate and complete behavioral data has typically always been difficult, laborious, and subjective, in recent years there has been rapid progress in methods for automatically quantifying behavior objectively and at scale. This progress has been primarily driven by the emergence of new computational hardware, software, and algorithms for measuring behavior. In order to reveal core insights about how animals organize their behavior with the increased quality and resolution of these data comes the need for new methods for data-driven modeling. Here, in this thesis, I focus on computational tools—the development of new algorithms and software—for measuring and modeling behavior using methods from computer vision, deep learning, Bayesian inference, and probabilistic programming, while also synthesizing these approaches with ideas from other relevant areas such as information theory, nonlinear dynamics, and statistical physics. First, I developed a barcode tracking system for automated behavioral studies where the location and identity of individuals can be reliably tracked over longer periods of time using conventional computer vision (Chapter 1). Next, I developed general-purpose deep learning methods for measuring animal posture --- any set of user-selected body parts --- in the laboratory and field (Chapter 2). Finally, I introduce methods for using these posture data to model behavior with techniques from machine learning and Bayesian statistical inference (Chapter 3). Together these methods reduce barriers to measuring and modeling animal behavior and allow researchers to answer scientific questions that were previously intractable.

	\chapter*{Zussamenfassung}
	\markboth{ZUSSAMENFASSUNG}{ZUSSAMENFASSUNG}
    \addcontentsline{toc}{chapter}{Zussamenfassung}
  Das Studium des Verhaltens von Tieren ist ein grundlegendes Bestreben zur Beantwortung wissenschaftlicher Fragen in einer Vielzahl von Bereichen --- einschließlich Neurowissenschaften, Psychologie, Ökologie, Genetik und Evolution. Während die Aufgabe, genaue und vollständige Verhaltensdaten zu sammeln, in der Regel immer schwierig, mühsam und subjektiv war, gab es in den letzten Jahren rasche Fortschritte bei den Methoden zur automatischen objektiven und maßstabsgetreuen Quantifizierung von Verhalten. Dieser Fortschritt wurde in erster Linie durch das Aufkommen neuer rechnergestützter Hardware, Software und Algorithmen zur Verhaltensmessung vorangetrieben. Mit der zunehmenden Qualität und Auflösung dieser Daten steigt auch der Bedarf an neuen Methoden zur datengesteuerten Modellierung, die zentrale Erkenntnisse darüber liefern können, wie Tiere ihr Verhalten organisieren. In dieser Arbeit konzentriere ich mich auf die Entwicklung neuer Algorithmen und Software zur Messung und Modellierung von Verhalten mit Methoden des Deep-Learning, der Bayes'schen Inferenz und der probabilistischen Programmierung, wobei ich diese Ansätze auch mit Ideen aus anderen relevanten Bereichen wie Informationstheorie, nichtlineare Dynamik und statistische Physik zusammenführe. Zunächst entwickelte ich in Zusammenarbeit mit Kollegen ein Strichcode-Verfolgungssystem für automatisierte Verhaltensstudien, bei dem der Aufenthaltsort und die Identität von Individuen über längere Zeiträume mit herkömmlicher Computer-Sicht zuverlässig verfolgt werden kann (Kapitel 1). Als nächstes entwickelte ich Mehrzweck-Deep-Learning-Methoden für die Messung der Verhaltung von Tieren --- jede Menge vom Benutzer ausgewählter Körperteile --- im Labor und vor Ort (Kapitel 2). Schließlich stelle ich Methoden zur Verwendung dieser Haltungsdaten zur Verhaltensmodellierung mit Techniken des maschinellen Lernens und der statistischen Bayes'schen Inferenz vor (Kapitel 3). Zusammen verringern diese Methoden die Barrieren bei der Messung und Modellierung des Verhaltens von Tieren und ermöglichen es den Forschern, wissenschaftliche Fragen zu beantworten, die zuvor unlösbar waren.
